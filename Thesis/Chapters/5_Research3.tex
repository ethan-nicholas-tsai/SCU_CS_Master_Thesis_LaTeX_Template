%!TEX root = ../Main.tex

\chapter{ 研究三}\label{ch-x}
\section{引言}
xx

\section{x}
\subsection{x}\label{sec-xx x}


\begin{table}[H]\footnotesize%[ht]\footnotesize
	\centering
	\begin{threeparttable}[b]

%	\caption{社xx准差}
	\bicaption{社交xxF1值}{Fxatasets}
	\label{tbl:adsfaf-detection}
	\begin{tabular}{lccccc}
		\hline
		\rowcolor{gray!15}
		方法 & 特征类型 & C-R-19\cite{conferencekey} & B-F-19\cite{conferencekey} & Twx0\cite{conferencekey} & Twx2\cite{conferencekey} \\
		\hline
		Dexot\cite{conferencekey} & 特xx本  & 51.2 ($\pm$0.0) & 77.0 ($\pm$0.0) & 73.1 ($\pm$0.0) & \underline{76.5} ($\pm$0.0) \\
		C-xM\cite{conferencekey} & 特x本  & 62.9 ($\pm$0.8) & \underline{74.0} ($\pm$4.7) & 59.6 ($\pm$0.7) & 65.9 ($\pm$0.0) \\
		MxM\cite{conferencekey} & x本  & / & / & 71.3 ($\pm$1.6) & 70.2 ($\pm$1.2) \\
		RxExa\cite{conferencekey} & x本  & / & / & 75.5 ($\pm$0.1) & 72.1 ($\pm$0.1) \\
		Tx\cite{conferencekey} & 文本  & / & / & 73.5 ($\pm$0.1) & 72.1 ($\pm$0.1) \\
		LxO\cite{conferencekey} & xxx本  & / & / & 77.4 ($\pm$0.2) & 75.7 ($\pm$0.1) \\
		\hline
	\end{tabular}
	\begin{tablenotes}
		\item[*] ``/''表示数据x应的模型。
	\end{tablenotes}
	\end{threeparttable}
\end{table}

嘻嘻嘻嘻

\begin{table}[htbp]\footnotesize
	\centering
	\bicaption{社x卷}{Sociaxuestionnaire}
	\label{tbl:trust survey}
	\begin{tabular}{|c|l|c|}
		\hline
		\rowcolor{gray!15}
		\multicolumn{3}{|l|}{\textbf{背}} \\
		\hline
		\multicolumn{3}{|p{14cm}|}{假设12324243424jifjoj9q9r8u0ufjfe0jfaeoua98ur9f9ebfcaeybfcyae87rca87fcasycacyfa7yc8ysy联系。} \\
		\hhline{*{3}{:=}:}
		\rowcolor{gray!15}
		\textbf{编x} & \;\;\,\textbf{问x述} & \textbf{选x} \\
		\hline
		Q1 & \;\;\,该用x确 & \begin{tabular}{l} 
			0\%(强x意), 10\%, 20\%, 30\%, 40\% \\
			50\%(不x), 60\%, 70\%, 80\%, 90\%, 100\%
		\end{tabular} \\
		\hline
		Q2 & \;\;\,该x知识 &  同Q1 \\
		\hline
		Q3 & \;\;\,此人xx欢迎 & 同Q1  \\
		\hline
		Q4 & \begin{tabular}{l} 
			该x些信息(如x介、\\x文等)  
			影响了x
		\end{tabular} 
		& [文x框] \\
		\hline
		Q5 & \;\;\,你是x择x该x户 & x注 / 忽x略 \\
		\hline
	\end{tabular}
\end{table}





\section{本章小结}

本章研究了xx。




