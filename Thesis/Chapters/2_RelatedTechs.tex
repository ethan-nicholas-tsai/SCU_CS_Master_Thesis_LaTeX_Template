%!TEX root = ../Main.tex

\chapter{相关技术介绍}
这章写多点哈
\section{引言}
本章将介绍xxx研究中涉及到的关键理论和技术,主要包括了xxx、xxx、xxx、xxx、xxx技术以及xxx技术。下面将对这些技术进行详细介绍。%下面将详细介绍各类技术的关键概念和相关原理。

\section{理论技术一}
\subsection{xxx}
迈尔斯-布里格斯类型指标(Myers-Briggs Type Indicator,简称MBTI)是一种深受欢迎且极具影响力的心理测评工具,它通过四个基本的二元维度——能量定向、信息获取、决策制定和生活方式——将个体分为十六种性格类型,每个类型代表了这些维度不同组合下的独特人格特征:
\begin{itemize}
	\item 能量定向:外倾(E)与内倾(I)
	\item 信息获取:实感(S)与直觉(N)
	\item 决策制定:思考(T)与情感(F)
	\item 生活方式:判断(J)与理解(P)
\end{itemize}

\section{统计学知识}
\subsection{斯皮尔曼相关系数}
斯皮尔曼相关系数(Spearman Correlation Coefficients)的计算方法如下:
	
\begin{enumerate}
	\item 为两个变量的每个观测值分配等级;
	\item 对每一对观测值,计算其等级差的平方,记为$D_i^2$;
	\item 计算所有$D_i^2$的和,记为$\sum D_i^2$;
	\item 使用以下公式计算斯皮尔曼相关系数$\rho$:
	 \begin{equation}
	 	\rho = 1 - \frac{6 \times \sum D_i^2}{n(n^2 - 1)}
	 \end{equation}
	其中,$n$是数据集中观测值的数量。
\end{enumerate}



\section{本章小结}
本章对xx所涉及的相关技术进行了详细的描述。首先介绍了xx。其次,介绍了xxx。最后,本章对xx做了必要的介绍。


