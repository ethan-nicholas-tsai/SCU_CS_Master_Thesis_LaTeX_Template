%!TEX root = ../Main.tex

% 作者在读期间科研成果简介
\chapter{攻读学位期间科研及成果简介}  % 攻读学位期间取得的研究成果
%\section*{发表论文}
%\emph{(暂无)}
%\section*{承担科研项目}
%\emph{(暂无)}

\noindent\textbf{(一)作者发表的学术论文情况}
\vspace{0.25em}

\noindent[1] ***, ***, ***, ***, ***. ***Social Network***[J]. \textit{Expert Systems with Applications}, 2023, 217: ***.(中科院1区期刊,已正式发表,本人第一作者)

\noindent[2] ***, ***, ***, ***, ***, ***. ***Coordinated Attack***[J]. \textit{IEEE Transactions on Information Forensics and Security}, 2024.(CCF-A期刊,第二轮评审中,本人第一作者)

\noindent[3] ***, ***, ***, ***, ***, ***, ***. ***Social Bot***[J]. \textit{IEEE Transactions on Knowledge and Data Engineering}, 2024.(CCF-A期刊,已投稿,本人第三作者)

\vspace{1em}

\noindent\textbf{(二)作者申请的专利情况}
\vspace{0.25em}

\noindent[1] ***,***,***,***,***,***,***. ***社交网络用户***[P]. 中国专利,申请专利号:ZL 2022***,申请时间:2022-**-**. (实审中,本人第二发明人,导师第一发明人)

\noindent[2] ***,***,***,***,***,***,***. ***社交机器人***[P]. 中国专利,申请专利号:ZL 2023***,申请时间:2023-**-**. (实审中,本人第三发明人)

\noindent[3] ***,***,***,***,***,***,***. ****谣言***[P]. 中国专利,专利号:ZL 2022***,授权时间:2022-**-**(已正式授权,本人第六发明人)

\vspace{1em}

\noindent\textbf{(三)研究生阶段参与的项目与课题}
\vspace{0.25em}

\noindent[1] 国家课题,项目名称:全称,项目编号:可以给全部的,起止日期:不需要脱敏,纵向项目,在研,参与.


\noindent[2] 四川省科划项目,项目名称:全称,项目编号:20x45,起止日期:202x.1-202x.12,纵向项目,在研,参与.


\noindent[3] 教育部x金项目,项目名称:网络x及应x究,项目编号:CM09,起止日期:2021.1-2021.12,纵向项目,已结题,参与.



