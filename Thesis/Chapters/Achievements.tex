%!TEX root = ../Main.tex

% 作者在读期间科研成果简介
\chapter{攻读学位期间科研及成果简介}  % 攻读学位期间取得的研究成果
%\section*{发表论文}
%\emph{(暂无)}
%\section*{承担科研项目}
%\emph{(暂无)}

\noindent\textbf{(一)作者发表的学术论文情况}
\vspace{0.25em}

\noindent[1] First Name Last Name, First Name Last Name, First Name Last Name. Paper Title [J]. \textit{Expert Systems with Applications}, 202x, 2x7: 11xx38.(中科院1区期刊,已正式发表,本人第一作者)

\noindent[2] First Name Last Name, First Name Last Name, First Name Last Name. Paper Title [J]. \textit{IEEE Transactions on Information Forensics and Security}, 2024.(CCF-A期刊,第二轮评审中,本人第一作者)

\noindent[3]x. x [J]. \textit{IEEE Transactions on Knowledge and Data Engineering}, 2024.(CCF-A期刊,已投稿,本人第三作者)

\vspace{1em}

\noindent\textbf{(二)作者申请的专利情况}
\vspace{0.25em}

\noindent[1] 中文名字,中文名字,中文名字,中文名字,中文名字,中文名字,中文名字. 一种x方法[P]. 中国专利,申请专利号:ZL 20x5.7,申请时间:202x-0x-02. (实审中,本人第二发明人,导师第一发明人)

\noindent[2] 中文名字,中文名字,中文名字,中文名字,中文名字,中文名字,中文名字. 基于多x方法[P]. 中国专利,专利号:ZL x.6,申请时间:202x-09-x2. (实审中,本人第三发明人)

\noindent[3] 中文名字,中文名字,中文名字,中文名字,中文名字,中文名字,中文名字. 一种基于x方法[P]. 中国专利,专利号:ZL 202x71266.1 授权时间:202x-07-0x.(已正式授权,本人第六发明人)

\vspace{1em}

\noindent\textbf{(三)研究生阶段参与的项目与课题}
\vspace{0.25em}

\noindent[1] 国家重x题,项目名称:智xx会x多x及实x系模型,项目编号:202xFx3101-2,起止日期:202x.1x-202x.0x,纵向项目,在研,参与.


\noindent[2] 四川x划项目,项目名称:面向网x协x技术研究,项目编号:202xG0145,起止日期:202x.1-202x.12,纵向项目,在研,参与.


\noindent[3] 教育部x项目,项目名称:网络x及应x研究,项目编号:CMx409,起止日期:202x.1-202x.12,纵向项目,已结题,参与.




