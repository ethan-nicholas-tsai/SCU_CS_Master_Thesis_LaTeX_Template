\chapter{pkuthss~文档类提供的功能}
	\section{文档类选项}

	pkuthss~文档类以~ctexbook~文档类为基础,
	其接受的所有文档类选项均被传递给~ctexbook。

	例如,ctexbook~文档类默认使用~GBK~编码,
	因此如果需要使用~UTF-8~编码撰写论文,
	则在导入~pkuthss~文档类时加上~\verb|UTF8|~选项即可:
\begin{verbatim}
      \usepackage[UTF8,...]{pkuthss} % “...”代表其它的选项。
\end{verbatim}
	类似地,如果需要使用~hyperref~宏包,
	则为了利用~ctexbook~文档类对~hyperref~宏包的支持特性,
	可以传递~\verb|hyperref|~选项给~ctexbook~宏包:
\begin{verbatim}
      \usepackage[hyperref,...]{pkuthss} % “...”代表其它的选项。
\end{verbatim}

	\section{pkuthss~文档类定义的命令}
		\subsection{用于设定文档信息的命令}

		这一类命令的语法是
\begin{verbatim}
      \commandname{具体信息} % commandname 为具体命令的名称。
\end{verbatim}

		这些命令总结如下\footnote%
		{\ \ttfamily\songti%
			\string\title、\string\author~和~\string\date~%
			实际上是从~ctexbook~文档类继承来的。%
		}:
		\begin{itemize}\denseenum
			\item \verb|\title|:设定论文标题;
			\item \verb|\etitle|:设定论文英文标题;
			\item \verb|\author|:设定作者;
			\item \verb|\eauthor|:设定作者的英文名;
			\item \verb|\date|:设定日期;
			\item \verb|\studentid|:设定学号;
			\item \verb|\school|:设定学院;
			\item \verb|\major|:设定专业;
			\item \verb|\emajor|:设定专业的英文名;
			\item \verb|\direction|:设定研究方向;
			\item \verb|\mentor|:设定导师;
			\item \verb|\ementor|:设定导师的英文名;
			\item \verb|\keywords|:设定关键词;
			\item \verb|\ekeywords|:英文关键词。
		\end{itemize}

		例如,如果要设定专业为“化学”,则可以使用以下命令:
\begin{verbatim}
      \major{化学}
\end{verbatim}

		\subsection{“\texttt{name}”类命令}

		这一类命令的语法是
\begin{verbatim}
      % commandname 为具体的命令名。
      \renewcommand{\commandname}{具体信息}
\end{verbatim}

		这些命令总结如下\footnote%
		{\ \ttfamily\songti%
			\string\contentsname~和~\string\bibname~%
			实际上是从~ctexbook~文档类继承来的。%
		}:
		\begin{itemize}\denseenum
			\item \verb|\thesisname|:论文类别名。
			\item \verb|\cabstractname|:中文摘要的标题。
			\item \verb|\eabstractname|:英文摘要的标题。
			\item \verb|\contentsname|:目录的标题。
			\item \verb|\bibname|:参考文献目录的标题。
		\end{itemize}

		例如,如果要设定论文的类别为“本科生毕业论文”,
		则可以使用以下命令:
\begin{verbatim}
      \renewcommand{\thesisname}{本科生毕业论文}
\end{verbatim}
		而如果要设定中文摘要的标题为“摘\hspace{2em}要”,
		则可以使用以下命令:
\begin{verbatim}
      \renewcommand{\abstractname}{摘\hspace{2em}要}
\end{verbatim}

		\subsection{其它命令}

		\begin{itemize}\denseenum
			\item \verb|\maketitle|:
				此命令根据设定好的文档信息自动生成论文的标题页,亦即封面。
			\item \verb|\specialchap|:
				此命令用于开始不进行标号但计入目录的一章,
				并合理安排其页眉。%
				\emph
				{%
					注意在此章内的节或小节等命令应使用带星号的版本,
					例如~\texttt{\string\section\string*}~等,
					以免造成章节编号混乱。%
				}
				\par
				例如,本文档中的“绪言”一章就是用~\verb|\specialchap{绪言}|~%
				这条命令开始的。
		\end{itemize}

		\subsection{从其它文档类和宏包继承的命令\label{sec:inherit}}

		pkuthss~文档类以~ctexbook~文档类为基础,
		并默认调用了以下宏包:
		\begin{itemize}\denseenum
			\item fntef:
				提供了~\verb|\maketitle|~中调用的~%
				\verb|\CJKunderline|~命令。
			\item graphicx\supercite{graphicx-doc}:提供图形支持。
			\item geometry\supercite{geometry-doc}:用于设置页面布局。
			\item fancyhdr\supercite{fancyhdr-doc}:用于设置页眉、页脚。
		\end{itemize}
		因此,ctexbook~文档类和这些宏包所提供的命令均可以使用。

		\vspace{1em}\par
		\emph
		{%
			注意:
			pkuthss~文档类中有一些一旦改动就有可能破坏预设的排版规划,
			因此不建议更改这些设置,它们是:
			\begin{itemize}\denseenum
				\item 纸张类型:A4;
				\item 版心尺寸:%
					$240\,\mathrm{mm}\times150\,\mathrm{mm}$,
					包含页眉、页脚;
				\item 默认字号:小四号。
			\end{itemize}
		}

	\section{pkuthss~文档类定义的环境}

	pkuthss~文档类定义了两个环境——\verb|cabstract|~和~\verb|cabstract|,
	分别用于编写中文和英文摘要。
	用户只需要写摘要的正文;标题、作者、导师、专业等部分会自动生成。

	此外,pkuthss~文档类还从第~\ref{sec:inherit}~节中所述的%
	文档类和宏包中继承了各种环境,用户也可以使用它们。

